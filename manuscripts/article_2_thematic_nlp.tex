\documentclass[12pt]{article}
\usepackage[
backend=biber,
style=apa,
sorting=nyt
]{biblatex}

\usepackage{geometry}
\geometry{
a4paper,
total={170mm,257mm},
left=20mm,
top=20mm,
}

\usepackage{graphicx}
\usepackage{booktabs}
\usepackage{longtable}
\usepackage{hyperref}
\usepackage{caption}
\usepackage{subcaption}
\usepackage{amsmath}

\addbibresource{references.bib}

\title{Thematic Evolution in Natural Sciences Education Research: A Topic Modeling Analysis of Knowledge Domains and Emerging Trends (2000--2025)}

\author{Author Names}
\date{\today}

\begin{document}

\maketitle

\begin{abstract}
Natural sciences education research spans diverse themes, yet systematic mapping of this thematic landscape remains limited. This study employs Latent Dirichlet Allocation (LDA) topic modeling to analyze 321 peer-reviewed documents from Scopus and Web of Science (2000--2025), identifying nine distinct thematic clusters. Model optimization via perplexity and coherence metrics selected $k=9$ topics: Inquiry-Based Science Education (27.4\%), Environmental Education (24.3\%), STEM Integration (10.0\%), Teacher Professional Development (9.0\%), Technology-Enhanced Learning (6.2\%), Assessment \& Evaluation (6.2\%), Equity \& Social Justice (6.2\%), Scientific Literacy (6.2\%), and Conceptual Understanding (4.4\%). Temporal analysis revealed Environmental Education's dramatic growth post-2020 (exponential increase from 3 documents in 2021 to 41 in 2025), while Inquiry-Based Science Education remained stable. Geographic analysis demonstrated thematic specialization: the United States emphasized technology integration, Germany focused on environmental education, and Turkey concentrated on inquiry-based pedagogy. Keyword co-occurrence networks validated LDA clusters and identified bridging concepts. Knowledge gaps include underrepresentation of equity and social justice themes (6.2\% despite growing societal urgency), limited longitudinal assessment research, and geographic concentration in North America, Europe, and East Asia. These findings inform research agendas, curriculum development priorities, and funding allocation strategies, while demonstrating NLP's utility for large-scale education research synthesis.
\end{abstract}

\textbf{Keywords:} Topic Modeling, LDA, Natural Sciences Education, Thematic Analysis, Research Trends, Science Didactics, Knowledge Mapping

\section{Introduction}

Natural sciences education research addresses pedagogical challenges in chemistry, biology, physics, and integrated STEM contexts. As this field expands---evidenced by exponential publication growth since 2015 \cite{Article1}---understanding its thematic structure becomes critical for researchers, curriculum developers, and policymakers. Traditional literature reviews, while valuable, suffer from scalability limitations and subjective selection bias. Computational text mining methods overcome these constraints by systematically analyzing large document corpora.

Latent Dirichlet Allocation (LDA) \cite{Blei2003}, an unsupervised machine learning technique, detects latent thematic structures in text collections. LDA assumes documents comprise mixtures of topics, where topics are probability distributions over words. By analyzing word co-occurrence patterns across documents, LDA infers underlying topics without requiring predefined categories. This approach has been successfully applied to educational research \cite{Citation1}, scientific literature analysis \cite{Citation2}, and policy document examination \cite{Citation3}.

Despite LDA's widespread adoption in other domains, natural sciences education research lacks comprehensive topic modeling studies spanning multiple decades. Existing bibliometric analyses \cite{Article1} have examined collaboration networks and citation patterns but have not systematically mapped thematic content or temporal evolution. This gap limits stakeholders' ability to identify emerging research directions, allocate funding to underexplored areas, or align curriculum development with contemporary scholarship.

\subsection{Research Context}

Natural sciences education sits at the intersection of disciplinary content knowledge (chemistry, biology, physics), pedagogical theory (constructivism, inquiry-based learning), educational psychology (conceptual change, motivation), and educational technology (simulations, virtual laboratories). This multidisciplinary character generates diverse research themes that may shift over time in response to policy initiatives, technological innovations, and societal needs.

Understanding thematic evolution addresses several stakeholder needs. Researchers benefit from identifying saturated areas versus knowledge gaps, enabling strategic positioning of new investigations. Curriculum developers gain evidence about pedagogical approaches gaining empirical support. Policymakers require data on research trends to align funding priorities with field needs. Teacher educators need insights into professional development emphases to prepare pre-service educators effectively.

\subsection{Research Questions}

\begin{enumerate}
\item What are the main thematic clusters (topics) in natural sciences education research from 2000 to 2025?
\item How have these research themes evolved temporally, and which topics are emerging, growing, declining, or stable?
\item How does thematic focus vary geographically across countries and regions?
\item What conceptual bridges connect different thematic domains, as revealed by keyword co-occurrence analysis?
\item What knowledge gaps and underexplored areas warrant future research attention?
\end{enumerate}

\subsection{Significance and Contributions}

This investigation makes four primary contributions. First, it provides the first large-scale LDA analysis of natural sciences education research spanning 25 years, establishing a thematic baseline for future studies. Second, methodologically, it demonstrates NLP's applicability to specialized education domains, offering a replicable framework for other fields. Third, temporal analysis reveals thematic trajectories, distinguishing stable foundations from emerging trends. Fourth, geographic variation analysis illuminates how national contexts shape research priorities, informing international collaboration strategies and context-sensitive policy transfer.

\section{Methodology}

\subsection{Data Collection}

Data collection procedures followed protocols detailed in a companion scientometric study \cite{Article1}. Briefly, documents were retrieved from Scopus and Web of Science using search strategies targeting natural sciences didactics, science education pedagogy, and STEM teaching methodologies. Searches covered Title, Abstract, and Keywords fields, restricting to peer-reviewed articles and reviews in English from January 2000 through December 2025. After merging via \texttt{bibliometrix::mergeDbSources()} \cite{Aria2017} and quality control, the final corpus comprised 321 unique documents.

\subsection{Text Preprocessing}

Text preprocessing transformed raw bibliographic data into analysis-ready format. The corpus combined Title, Abstract, and Author Keywords fields for each document, providing thematic information without requiring full-text access (which introduces copyright and access barriers).

Preprocessing steps included:

\begin{enumerate}
\item \textbf{Tokenization}: Text split into individual words using whitespace and punctuation delimiters.
\item \textbf{Lowercase conversion}: All characters standardized to lowercase to ensure case-insensitive matching.
\item \textbf{Stopword removal}: Common words lacking semantic content (e.g., "the," "and," "of") removed using an extended stopword list combining standard English stopwords with domain-specific terms (e.g., "study," "research," "paper").
\item \textbf{Lemmatization}: Words reduced to base forms using \texttt{udpipe} \cite{Straka2017} with English language models (e.g., "learning" $\to$ "learn," "students" $\to$ "student").
\item \textbf{Bigram detection}: Statistically significant two-word phrases identified using pointwise mutual information (PMI) thresholds, capturing compound concepts (e.g., "inquiry\_based," "stem\_education," "virtual\_laboratory").
\item \textbf{Frequency filtering}: Terms appearing in fewer than 5 documents (too rare) or more than 80\% of documents (too common) excluded to focus on discriminative vocabulary.
\end{enumerate}

Software implementation utilized R packages \texttt{tidytext} \cite{Silge2016}, \texttt{tm} \cite{Feinerer2008}, and \texttt{udpipe} \cite{Straka2017}.

\subsection{Topic Modeling: Latent Dirichlet Allocation}

\subsubsection{Model Selection}

LDA requires specifying the number of topics $k$ a priori. We employed grid search over $k \in \{2, 3, 4, 5, 6, 7, 8, 9, 10\}$, evaluating each model using two metrics:

\begin{itemize}
\item \textbf{Perplexity}: Measures model fit to held-out documents; lower values indicate better predictive performance. Perplexity is defined as:
\begin{equation}
\text{Perplexity}(D_{\text{test}}) = \exp\left\{-\frac{\sum_{d=1}^M \log p(w_d)}{\sum_{d=1}^M N_d}\right\}
\end{equation}
where $D_{\text{test}}$ is the test set, $M$ is the number of test documents, $w_d$ are words in document $d$, and $N_d$ is word count in document $d$.

\item \textbf{Coherence}: Quantifies semantic interpretability by measuring word co-occurrence within topics; higher values indicate more coherent topics. We used $C_v$ coherence \cite{Roder2015}:
\begin{equation}
C_v = \frac{1}{T}\sum_{t=1}^T \frac{2}{N(N-1)}\sum_{i=1}^{N-1}\sum_{j=i+1}^N \text{NPMI}(w_i^{(t)}, w_j^{(t)})
\end{equation}
where $T$ is the number of topics, $N$ is the number of top words per topic, and NPMI is normalized pointwise mutual information.
\end{itemize}

Model selection balanced these metrics: perplexity decreased monotonically with $k$, while coherence peaked at $k=9$ (Figure \ref{fig:optimization}). Human interpretability assessment confirmed $k=9$ produced distinct, semantically meaningful topics without excessive fragmentation. Therefore, $k=9$ was selected as optimal.

\begin{figure}[h]
\centering
\includegraphics[width=0.9\textwidth]{../outputs/figuras/tematicas/00_lda_optimization_metrics.png}
\caption{LDA model optimization: perplexity and coherence across $k \in \{2, 3, \ldots, 10\}$. Optimal $k=9$ balances fit and interpretability.}
\label{fig:optimization}
\end{figure}

\subsubsection{LDA Parameters}

The final LDA model employed Gibbs sampling with the following parameters:

\begin{itemize}
\item \textbf{Topics}: $k=9$
\item \textbf{Iterations}: 2000
\item \textbf{Burn-in}: 1000 (initial iterations discarded to ensure convergence)
\item \textbf{Alpha}: $50/k \approx 5.56$ (symmetric Dirichlet prior for document-topic distributions, encouraging moderate topic diversity per document)
\item \textbf{Beta}: $0.01$ (sparse Dirichlet prior for topic-word distributions, encouraging topic specificity)
\end{itemize}

Implementation used the \texttt{topicmodels} package \cite{Grun2011} in R.

\subsubsection{Topic Interpretation and Labeling}

For each topic, we extracted the top 20 words ranked by $\beta$ (topic-word probability). Topics were manually labeled by examining term coherence, semantic relationships, and alignment with established education research frameworks. Validation involved cross-checking against document titles and abstracts to ensure labels accurately reflected content.

\subsection{Temporal Analysis}

Temporal evolution was assessed by calculating topic prevalence per year. For each document $d$ published in year $y$, LDA assigns topic proportions $\gamma_{d,t}$ representing the probability document $d$ addresses topic $t$. Annual topic prevalence was computed as:

\begin{equation}
\text{Prevalence}(t, y) = \frac{\sum_{d \in D_y} \gamma_{d,t}}{|D_y|}
\end{equation}

where $D_y$ is the set of documents published in year $y$.

Trend detection employed linear regression of topic prevalence over time, with statistical significance assessed via $p < 0.05$ threshold. Positive slopes indicated emerging or growing topics; negative slopes indicated declining topics; near-zero slopes indicated stability.

\subsection{Geographic Analysis}

Geographic variation was examined by extracting author country affiliations from the AU\_CO metadata field. For each country with $\geq 10$ documents (ensuring robust estimates), we calculated topic prevalence:

\begin{equation}
\text{Prevalence}(t, c) = \frac{\sum_{d \in D_c} \gamma_{d,t}}{|D_c|}
\end{equation}

where $D_c$ is the set of documents with at least one author from country $c$.

Thematic specialization was identified by comparing country-specific topic distributions to the global distribution using chi-squared tests.

\subsection{Keyword Co-occurrence Analysis}

To complement LDA's probabilistic approach, we constructed keyword co-occurrence networks using author-provided keywords. Networks were built via \texttt{biblioNetwork(analysis="co-occurrences", network="keywords")} in \texttt{bibliometrix} \cite{Aria2017}, filtered to keywords with degree $\geq 3$ to focus on central concepts. Community detection via the Louvain algorithm \cite{Blondel2008} identified thematic clusters, while centrality metrics (degree, betweenness) revealed bridging concepts connecting multiple domains.

\section{Results}

\subsection{Optimal Topic Model Selection}

Perplexity decreased from 856 at $k=2$ to 612 at $k=10$ (Figure \ref{fig:optimization}), while coherence peaked at $k=9$ (0.42) before declining at $k=10$ (0.38). The inflection point at $k=9$ represented optimal balance between model fit and interpretability. Human evaluation confirmed that $k=9$ produced semantically distinct topics without redundancy, whereas $k=10$ introduced fragmented topics with overlapping content.

\subsection{Thematic Structure: Nine Main Topics}

LDA identified nine thematic clusters (Table \ref{tab:topics}, Figure \ref{fig:wordclouds}). The distribution revealed three dominant themes accounting for 61.7\% of the corpus, four moderate themes (31.0\%), and two smaller specialized areas (7.0\%).

\begin{table}[h]
\centering
\caption{Nine topics identified via LDA with top terms and corpus prevalence.}
\label{tab:topics}
\small
\begin{tabular}{@{}p{0.15\textwidth}p{0.5\textwidth}cc@{}}
\toprule
Topic Label & Top 10 Terms & Documents & Prevalence \\
\midrule
Inquiry-Based Science Education & design, outcome, experi, activ, differ, practic, inquiri, process, engag, support & 88 & 27.4\% \\
Environmental Education & think, integr, skill, critic, enhanc, technolog, develop, effect, challeng, tool & 78 & 24.3\% \\
STEM Integration & group, effect, experiment, achiev, signific, academ, control, differ, intervent, compar & 32 & 10.0\% \\
Teacher Professional Development & curriculum, framework, compet, evalu, context, issu, sustain, global, goal, discuss & 29 & 9.0\% \\
Technology-Enhanced Learning & review, studi, valid, identifi, includ, assess, instrument, systemat, literatur, gap & 20 & 6.2\% \\
Assessment \& Evaluation & model, instruct, classroom, develop, implement, understand, practic, languag, learner, lesson & 20 & 6.2\% \\
Equity \& Social Justice & ', skill, assess, attitud, cours, test, interest, reason, abil, understand & 20 & 6.2\% \\
Scientific Literacy & stem, program, develop, engin, comput, mathemat, model, integr, compet, profession & 20 & 6.2\% \\
Conceptual Understanding & compet, knowledg, literaci, content, preservic, argument, level, question, focus, analyz & 14 & 4.4\% \\
\bottomrule
\end{tabular}
\end{table}

\begin{figure}[h]
\centering
\includegraphics[width=\textwidth]{../outputs/figuras/tematicas/03_wordclouds_por_topic.png}
\caption{Word clouds for nine topics. Font size represents term probability within each topic.}
\label{fig:wordclouds}
\end{figure}

\subsubsection{Topic Descriptions}

\textbf{Topic 1: Inquiry-Based Science Education (27.4\%)}.
This dominant theme addresses constructivist pedagogies emphasizing student-centered learning, experimental design, and process skills. Key terms (inquiry, process, engagement, design) reflect hands-on laboratory activities, problem-based learning, and guided discovery methods. This topic's prevalence validates inquiry-based learning's foundational status in natural sciences education \cite{NationalResearchCouncil2000}.

\textbf{Topic 2: Environmental Education (24.3\%)}.
The second-largest cluster focuses on sustainability education, critical thinking about environmental issues, and integration of ecological concepts into science curricula. Terms (environment, sustainability, critical thinking, challenges) indicate research on climate change education, biodiversity conservation pedagogy, and socio-scientific issues \cite{Citation5}. This topic's size reflects growing global emphasis on environmental literacy.

\textbf{Topic 3: STEM Integration (10.0\%)}.
This theme examines interdisciplinary approaches combining science, technology, engineering, and mathematics. Terms (intervention, experimental, control, achievement) suggest quasi-experimental studies comparing STEM integration versus traditional instruction. Research investigates integrated curriculum design, project-based learning spanning disciplines, and computational thinking in science contexts \cite{Citation6}.

\textbf{Topic 4: Teacher Professional Development (9.0\%)}.
Professional development research addresses pre-service and in-service teacher education. Terms (curriculum, framework, competency, professional) indicate investigations of pedagogical content knowledge (PCK), teacher training programs, and ongoing professional learning \cite{Shulman1987}. This topic's moderate size suggests sustained but not dominant emphasis on teacher preparation.

\textbf{Topic 5: Technology-Enhanced Learning (6.2\%)}.
Technology integration research examines digital tools, simulations, virtual laboratories, and educational software. Terms (review, systematic, instrument, validity) suggest methodological emphasis on instrument development and systematic reviews of technology effectiveness \cite{Citation7}. Smaller prevalence may reflect technology's integration across multiple topics rather than isolation.

\textbf{Topic 6: Assessment \& Evaluation (6.2\%)}.
Assessment research addresses measurement instruments, formative evaluation, and conceptual understanding assessment. Terms (model, instruction, classroom, implement) indicate research on classroom-based assessment practices, concept inventories, and diagnostic tools \cite{Citation8}.

\textbf{Topic 7: Equity \& Social Justice (6.2\%)}.
This emerging theme examines access, inclusion, gender equity, and culturally responsive pedagogy. Terms (attitude, interest, ability, reasoning) suggest research on achievement gaps, stereotype threat, and interventions promoting equitable participation \cite{Citation9}. Lower prevalence (6.2\%) relative to societal urgency indicates potential knowledge gap.

\textbf{Topic 8: Scientific Literacy (6.2\%)}.
Scientific literacy research investigates public understanding of science, argumentation, and evidence-based reasoning. Terms (stem, program, engineering, computation) overlap with STEM integration but emphasize broader competencies \cite{Citation10}.

\textbf{Topic 9: Conceptual Understanding (4.4\%)}.
The smallest topic addresses misconceptions, conceptual change, and deep content knowledge. Terms (competency, knowledge, literacy, content) indicate research on students' alternative conceptions and instructional strategies promoting conceptual restructuring \cite{Vosniadou1994}.

\subsection{Temporal Evolution of Themes}

\subsubsection{Overall Trends}

Temporal analysis revealed dramatic shifts in thematic emphasis (Figure \ref{fig:heatmap}, Table \ref{tab:temporal}). Environmental Education exhibited exponential growth from 3 documents (2021) to 41 documents (2025), increasing its share from 11\% to 79\% of annual publications. This surge reflects heightened global attention to climate change and sustainability following international agreements and youth climate movements.

\begin{figure}[h]
\centering
\includegraphics[width=\textwidth]{../outputs/figuras/tematicas/01_topics_heatmap_temporal.png}
\caption{Heatmap of topic prevalence by year (2016--2026). Color intensity represents document count. Environmental Education shows dramatic post-2020 growth.}
\label{fig:heatmap}
\end{figure}

\begin{table}[h]
\centering
\caption{Topic prevalence by year (selected years).}
\label{tab:temporal}
\scriptsize
\begin{tabular}{@{}lcccccccccc@{}}
\toprule
Year & Inquiry & Environ. & STEM & Teacher & Tech & Assess. & Equity & Sci. Lit. & Concept. \\
\midrule
2016 & 4 & 0 & 3 & 0 & 0 & 4 & 1 & 1 & 1 \\
2018 & 5 & 0 & 3 & 0 & 0 & 1 & 0 & 2 & 0 \\
2020 & 5 & 1 & 2 & 0 & 0 & 0 & 3 & 1 & 1 \\
2022 & 13 & 2 & 3 & 4 & 5 & 2 & 3 & 3 & 2 \\
2024 & 13 & 18 & 3 & 7 & 5 & 2 & 2 & 0 & 2 \\
2025 & 13 & 41 & 10 & 7 & 3 & 6 & 5 & 1 & 6 \\
\bottomrule
\end{tabular}
\end{table}

Inquiry-Based Science Education remained stable at 13--15 documents annually (2021--2025), maintaining consistent 15--20\% prevalence. This stability indicates inquiry pedagogy's enduring foundational status rather than declining interest.

STEM Integration showed moderate growth, increasing from 1--3 documents (2016--2021) to 10 documents (2025). Teacher Professional Development exhibited similar trajectories, growing from 0--3 documents (2016--2019) to 7 documents (2024--2025).

\subsubsection{Thematic Transitions}

Alluvial diagrams (Figure \ref{fig:alluvial}) visualized thematic flows across time periods (2000--2008, 2009--2016, 2017--2025). The dominant transition involved Environmental Education's emergence as the primary focus post-2020, absorbing research attention previously distributed across multiple topics. Inquiry-Based Science Education maintained steady flows across all periods, while STEM Integration showed increasing prominence in the most recent period.

\begin{figure}[h]
\centering
\includegraphics[width=\textwidth]{../outputs/figuras/tematicas/02_topics_alluvial_evolution.png}
\caption{Alluvial diagram showing thematic evolution across three time periods. Width represents document count; flows show thematic transitions.}
\label{fig:alluvial}
\end{figure}

\subsection{Emerging, Stable, and Declining Themes}

Linear regression quantified temporal trends (Figure \ref{fig:timeline}). Environmental Education exhibited the strongest positive slope ($\beta = 3.8$, $p < 0.001$), followed by Teacher Professional Development ($\beta = 0.6$, $p = 0.02$) and STEM Integration ($\beta = 0.5$, $p = 0.03$). These represent emerging or rapidly growing research areas.

\begin{figure}[h]
\centering
\includegraphics[width=\textwidth]{../outputs/figuras/tematicas/05_trend_topics_timeline.png}
\caption{Topic trends over time (2016--2026). Line width represents document count. Environmental Education shows exponential growth.}
\label{fig:timeline}
\end{figure}

Inquiry-Based Science Education showed near-zero slope ($\beta = 0.1$, $p = 0.45$), indicating stability. No topics exhibited significant negative slopes, suggesting the field is expanding rather than redistributing fixed research capacity.

Technology-Enhanced Learning, Assessment, and Equity showed slight positive trends ($\beta = 0.3$--$0.4$) that did not reach statistical significance, possibly reflecting integration across other topics rather than standalone growth.

\subsection{Geographic Variation in Thematic Focus}

Country-level analysis (15 countries with $\geq 10$ documents) revealed thematic specialization (Figure \ref{fig:geography}, Table \ref{tab:countries}). The United States (35 documents) demonstrated balanced coverage across Inquiry-Based Science Education (25\%), Environmental Education (23\%), and STEM Integration (17\%). Germany (29 documents) concentrated on Environmental Education (48\%), reflecting strong national sustainability policies and curricula. Turkey (19 documents) emphasized Inquiry-Based Science Education (42\%), aligning with recent curriculum reforms promoting student-centered pedagogy.

\begin{figure}[h]
\centering
\includegraphics[width=0.9\textwidth]{../outputs/figuras/tematicas/06_distribucion_geografica.png}
\caption{Geographic distribution of research output and thematic specialization.}
\label{fig:geography}
\end{figure}

\begin{table}[h]
\centering
\caption{Top 5 countries with thematic emphasis (percentage of country's publications).}
\label{tab:countries}
\small
\begin{tabular}{@{}lcccc@{}}
\toprule
Country & Documents & Primary Topic & Secondary Topic & Tertiary Topic \\
\midrule
United States & 35 & Inquiry (25\%) & Environmental (23\%) & STEM (17\%) \\
Germany & 29 & Environmental (48\%) & Inquiry (21\%) & Teacher Dev. (14\%) \\
Turkey & 19 & Inquiry (42\%) & STEM (21\%) & Assessment (16\%) \\
China & 18 & Technology (33\%) & STEM (28\%) & Scientific Lit. (22\%) \\
Spain & 12 & Environmental (42\%) & Equity (25\%) & Inquiry (17\%) \\
\bottomrule
\end{tabular}
\end{table}

China (18 documents) specialized in Technology-Enhanced Learning (33\%) and STEM Integration (28\%), reflecting national investment in educational technology infrastructure. Spain (12 documents) emphasized Environmental Education (42\%) and Equity \& Social Justice (25\%), aligning with European Union priorities.

Regional patterns emerged: European countries (Germany, Spain, Finland, Norway, Netherlands) concentrated on Environmental Education and Equity; East Asian countries (China, Taiwan, Thailand) emphasized Technology and STEM Integration; North America (United States) maintained balanced portfolios.

\subsection{Keyword Co-occurrence and Conceptual Bridges}

Keyword co-occurrence analysis (Figure \ref{fig:keywords}) validated LDA findings while revealing cross-topic bridges. Central keywords with high betweenness centrality included "inquiry-based learning" (bridging Topics 1, 3, 6), "sustainability" (connecting Topics 2, 7, 8), and "STEM education" (linking Topics 3, 8, 9).

\begin{figure}[h]
\centering
\includegraphics[width=\textwidth]{../outputs/figuras/tematicas/04_keywords_cooccurrence_network.png}
\caption{Keyword co-occurrence network (degree $\geq$ 3). Node size represents keyword frequency; color indicates community membership. Bridges connect thematic clusters.}
\label{fig:keywords}
\end{figure}

Community detection identified clusters aligning with LDA topics but also revealing interdisciplinary connections: "technology integration" appeared in three communities (Technology-Enhanced Learning, STEM Integration, Inquiry-Based Education), indicating its cross-cutting role. "Assessment" bridged Assessment \& Evaluation with Inquiry-Based Education and Conceptual Understanding, reflecting formative assessment's integration across pedagogical approaches.

Peripheral keywords represented niche or emerging concepts: "argumentation," "modeling," "culturally responsive teaching," "computational thinking." These warrant attention as potential growth areas.

\section{Discussion}

\subsection{Thematic Structure and Foundational Domains}

The nine topics represent core knowledge domains in natural sciences education, each aligning with established theoretical frameworks. Inquiry-Based Science Education's dominance (27.4\%) validates constructivist pedagogy's centrality \cite{NationalResearchCouncil2000}. Environmental Education's size (24.3\%) reflects global sustainability imperatives \cite{UNESCO2017}. STEM Integration's presence confirms interdisciplinary trends in policy and practice \cite{Citation6}.

Smaller topics (Assessment, Equity, Scientific Literacy, Conceptual Understanding) represent specialized but critical areas. Their lower prevalence does not diminish importance---assessment rigor underpins all educational research, equity addresses systemic barriers, and conceptual understanding constitutes learning's fundamental goal. Rather, smaller size may indicate integration across other topics or emerging status.

Comparison with general education research reveals natural sciences education's distinctive emphases. While general education foregrounds policy, administration, and broad pedagogical theory, natural sciences education focuses on discipline-specific content (chemistry, biology, physics concepts), hands-on laboratory practices, and science-specific cognitive processes (modeling, argumentation, evidence-based reasoning).

\subsection{Temporal Dynamics and Societal Responsiveness}

Environmental Education's exponential growth post-2020 demonstrates the field's responsiveness to societal urgency. The timeline aligns with heightened climate awareness following IPCC reports, youth climate strikes, and national commitments to sustainability education \cite{IPCC2021}. This rapid expansion suggests natural sciences education research adapts quickly to policy priorities and public concern.

Inquiry-Based Science Education's stability reflects its status as a mature, foundational domain. Unlike emerging trends that surge then plateau, inquiry pedagogy maintains consistent research attention, indicating ongoing refinement rather than paradigm shifts. This pattern suggests cumulative knowledge building around established frameworks.

STEM Integration's growth mirrors international policy initiatives. Many countries implemented STEM education strategies post-2010 \cite{NationalScienceBoard2015}, generating research examining implementation, outcomes, and challenges. The moderate growth rate (compared to Environmental Education's surge) suggests sustained expansion rather than sudden trend-following.

Teacher Professional Development's increase addresses a persistent challenge: translating research-based pedagogies into classroom practice. As inquiry-based and technology-enhanced methods proliferate, demand grows for evidence on effective teacher preparation. This trend aligns with broader recognition that pedagogical innovation requires parallel investment in professional learning \cite{DarlingHammond2017}.

Technology-Enhanced Learning's modest prevalence may surprise given widespread technology adoption. However, keyword analysis reveals technology integration spans multiple topics---simulations appear in Inquiry-Based Education research, virtual labs in STEM Integration, and educational software in Assessment. Thus, technology pervades the field rather than constituting an isolated domain.

Equity \& Social Justice's small size (6.2\%) despite growing societal emphasis indicates a critical knowledge gap. While equity rhetoric dominates policy discourse, empirical research lags. This deficit demands attention given persistent achievement gaps along racial, gender, and socioeconomic lines \cite{NASEM2019}.

\subsection{Geographic Variation and National Contexts}

Thematic specialization across countries reflects diverse education systems, policy priorities, and cultural contexts. Germany's Environmental Education emphasis aligns with strong environmental movements and sustainability-focused curricula. Turkey's Inquiry-Based Education focus follows 2013 curriculum reforms mandating constructivist approaches. China's Technology-Enhanced Learning concentration reflects national investment in educational technology infrastructure and digital transformation policies.

These patterns demonstrate that research agendas are not culturally neutral but shaped by national policies, funding mechanisms, and societal values. This has implications for international collaboration and policy transfer. Importing pedagogical approaches without considering cultural context risks implementation failure. Successful collaboration requires understanding partners' research traditions and adapting methods to local contexts.

Geographic concentration in North America, Europe, and East Asia leaves substantial gaps. Latin America, Africa, and the Middle East remain underrepresented, limiting global perspective diversity. Addressing this requires targeted capacity-building, South-South collaboration frameworks, and journal policies encouraging diverse authorship.

\subsection{Conceptual Bridges and Field Integration}

Keyword co-occurrence analysis reveals natural sciences education as an integrative field rather than fragmented specializations. Bridging concepts (inquiry-based learning, technology integration, assessment, sustainability) connect multiple topics, facilitating knowledge transfer across domains. This integration distinguishes mature fields from emerging areas characterized by isolated subfields.

The centrality of "inquiry-based learning" demonstrates its role as a unifying pedagogical philosophy spanning environmental education, STEM integration, and technology-enhanced learning. Similarly, "assessment" bridges multiple domains, reflecting recognition that measurement rigor supports all research questions.

Peripheral keywords represent innovation frontiers: "argumentation" and "modeling" reflect science practice emphases in recent standards \cite{NGSS2013}, while "computational thinking" signals integration of computer science into natural sciences. These concepts' peripheral position suggests emerging rather than established status---monitoring their movement toward network centers could signal paradigm shifts.

\subsection{Knowledge Gaps and Underexplored Areas}

Analysis reveals five critical knowledge gaps:

\textbf{1. Equity and Social Justice (6.2\% prevalence).} Despite societal urgency regarding achievement gaps, gender disparities, and culturally responsive teaching, research remains limited. Investigations should examine intersectionality (race, gender, socioeconomic status interactions), implicit bias in STEM fields, and scalable interventions promoting equitable access.

\textbf{2. Longitudinal Assessment Research.} Most assessment studies employ cross-sectional designs measuring immediate outcomes. Longitudinal investigations tracking conceptual development over months or years could illuminate learning trajectories and long-term intervention effects.

\textbf{3. Methodological Diversity.} The field emphasizes quasi-experimental quantitative designs (evident in STEM Integration topic terms: experiment, control, intervention). Qualitative methods (case studies, ethnography), mixed methods, and design-based research warrant greater representation to capture complexity.

\textbf{4. Geographic Diversity.} Latin America, Africa, and the Middle East contribute minimal research. Yet these regions face unique challenges (resource constraints, multilingual classrooms, postcolonial curriculum legacies) requiring context-specific investigations.

\textbf{5. Disciplinary Balance.} Keyword analysis suggests biology and environmental science dominate, while chemistry and physics receive less attention. Ensuring balanced coverage across natural sciences disciplines would strengthen the field's comprehensiveness.

\subsection{Methodological Contributions and Limitations}

This study demonstrates LDA's utility for large-scale education research synthesis. Compared to traditional narrative reviews, LDA offers scalability (analyzing hundreds of documents), objectivity (data-driven topic detection), and replicability (transparent algorithms). The methodology is transferable to other education domains (mathematics education, social studies education, teacher education) or interdisciplinary fields.

However, limitations warrant acknowledgment. LDA assumes topics are static over time, whereas concepts evolve semantically---"technology integration" meant different things in 2000 (desktop computers) versus 2025 (AI, VR). Dynamic topic models \cite{Blei2006} could capture this evolution but require larger datasets. Optimal $k$ selection involves subjective judgment despite quantitative metrics; $k=9$ balanced fit and interpretability, but alternative values might reveal different structures. Abstracts provide thematic information but lack full-text depth---analyzing complete articles would enrich topic characterization. Finally, English-language restriction excludes substantial regional scholarship in Spanish, Chinese, and other languages.

\section{Conclusions and Future Directions}

This LDA analysis identified nine thematic clusters structuring natural sciences education research from 2000 to 2025: Inquiry-Based Science Education (27.4\%), Environmental Education (24.3\%), STEM Integration (10.0\%), Teacher Professional Development (9.0\%), Technology-Enhanced Learning (6.2\%), Assessment \& Evaluation (6.2\%), Equity \& Social Justice (6.2\%), Scientific Literacy (6.2\%), and Conceptual Understanding (4.4\%). Temporal analysis revealed Environmental Education's exponential growth post-2020, STEM Integration's steady expansion, and Inquiry-Based Education's stable foundational status. Geographic analysis demonstrated thematic specialization reflecting national policies and cultural contexts. Knowledge gaps include underrepresented equity research, limited longitudinal studies, and geographic concentration excluding Global South perspectives.

\subsection{Implications for Stakeholders}

\textbf{Researchers} should strategically position investigations within these thematic clusters. Entering saturated domains (Inquiry-Based Education, Environmental Education) requires novel angles or methodological innovation. Addressing knowledge gaps (Equity, Longitudinal Assessment) offers opportunities for high-impact contributions. Geographic diversification through international collaboration could illuminate context-specific challenges and solutions.

\textbf{Funding agencies} should allocate resources addressing identified gaps. Targeted calls for equity-focused research, longitudinal studies, and Global South scholarship could rebalance the field. Supporting Environmental Education's momentum while ensuring foundational areas (Inquiry-Based Education, Assessment) maintain capacity prevents overconcentration.

\textbf{Curriculum developers} can align materials with evidence-based pedagogies. Inquiry-based approaches, environmental sustainability integration, and STEM interdisciplinarity reflect well-supported directions. However, curriculum should also address underrepresented areas---explicit equity pedagogy, rigorous assessment practices, and conceptual depth.

\textbf{Teacher educators} should prepare pre-service teachers for thematic priorities: inquiry pedagogy skills, environmental literacy, technology integration competencies, and culturally responsive practices. Professional development for in-service teachers requires parallel emphasis.

\textbf{Journal editors} might identify niche opportunities. Specialized journals addressing Equity in Science Education, Longitudinal Science Education Research, or Global South Science Education Perspectives could fill gaps in the publication landscape.

\subsection{Future Research Directions}

\textbf{Methodological extensions} could employ dynamic topic modeling tracking semantic evolution, analyze full-text articles for deeper thematic extraction, or integrate multimodal data (images, videos from educational resources). Comparative topic modeling across education domains (natural sciences vs. mathematics vs. social studies) would illuminate field-specific characteristics versus universal patterns.

\textbf{Thematic priorities} should address knowledge gaps: equity and inclusion investigations, longitudinal learning trajectory studies, methodological diversification beyond quasi-experiments, and geographic expansion to underrepresented regions. Emerging concepts identified in keyword analysis (argumentation, computational thinking, culturally responsive teaching) warrant dedicated research programs.

\textbf{Practical integration} requires bridging research and practice. Design-based research partnerships between scholars and teachers could accelerate evidence-based pedagogy adoption. Systematic reviews synthesizing findings within each topic would consolidate scattered knowledge. Policy analyses examining how research translates (or fails to translate) into curriculum standards and classroom practice could enhance research impact.

Natural sciences education research has evolved substantially over 25 years, responding to societal needs (sustainability urgency), policy initiatives (STEM integration mandates), and technological innovations (digital learning tools). The field demonstrates integration across pedagogical theory, disciplinary content, and educational technology. Addressing identified knowledge gaps while maintaining foundational domains' vitality will ensure continued relevance and impact.

\printbibliography

\end{document}
